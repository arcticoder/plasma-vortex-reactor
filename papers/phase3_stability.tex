% author_config.tex (gitignored)
\newcommand{\authorname}{Ryan Sherrington}
\newcommand{\authoremail}{ryan.sherrington@gmx.co.uk}
\documentclass[11pt]{article}
\usepackage{graphicx}
\usepackage{hyperref}
\title{Plasma Vortex Reactor: CI-validated Modeling with Stability KPI Gates and Dynamic Ripple Control}
\author{See author\_config.tex}
\date{\today}
\begin{document}
\maketitle
\begin{abstract}
We present a reproducible modeling and validation pipeline for a plasma vortex reactor with CI-enforced KPI gates, dynamic ripple control, and artifact-rich reporting.
\end{abstract}
\section{Introduction}
Motivation and contributions.
\section{Methods}
Outline of reactor model, dynamic ripple, and CI.
\section{Results}
\subsection{Operating Envelope}
Using the physics-based yield proxy (Sec.~Methods), we sweep density $n$ and electron temperature $T$ to chart an operating envelope of the Figure of Merit (FOM). The contour in Fig.~\ref{fig:envelope} shows a monotonic increase of FOM with $T$ at fixed $n$ over the range $10\,\mathrm{eV}$ to $30\,\mathrm{eV}$, consistent with our golden test. The top-$k$ frontier points extracted from the envelope emphasize regions near $n\approx 10^{20}\,\mathrm{cm}^{-3}$ and $T\gtrsim 25\,\mathrm{eV}$ where FOM peaks.

\subsection{KPI Gates and Ripple}
Across CI runs, feasibility gates report stable operation (\texttt{stable=true}) for baseline scenarios and intentionally fail under high ripple. The dynamic stability vs ripple plot (Fig.~\ref{fig:ripple}) shows the expected degradation with increasing RMS ripple. The CI comment includes a KPI delta table comparing current FOM to the baseline with a performance budget badge. In our latest run, we observe $\Delta\mathrm{FOM}\approx 0.00$ (no regression) and a green budget badge (elapsed well under 2.0\,s).

\subsection{Time-to-Stability and Yield}
Time-to-stability and time-to-yield markers are overlaid onto the envelope context via a dual-panel figure to facilitate scenario selection. Typical stabilization occurs within $\mathcal{O}(10^2)$\,ms for the synthetic sweeps, while energy traces remain within budget constraints.
\section{Reproducibility}
Bundle, schemas, and CI.
\section{Conclusion}
Future work.
\begin{figure}[h]
\centering
\includegraphics[width=0.7\linewidth]{operating_envelope.png}
\caption{Operating envelope (FOM vs $n$, $T$).}\label{fig:envelope}
\end{figure}
\begin{figure}[h]
\centering
\includegraphics[width=0.6\linewidth]{dynamic_stability_ripple.png}
\caption{Stability vs ripple with gate.}\label{fig:ripple}
\end{figure}
\begin{figure}[h]
\centering
\includegraphics[width=0.6\linewidth]{kpi_trend.png}
\caption{KPI trend (FOM across revisions).}\label{fig:kpi-trend}
\end{figure}
\begin{figure}[h]
\centering
\includegraphics[width=0.8\linewidth]{envelope_dual_panel.png}
\caption{Dual-panel: envelope (left) with FOM heat and time-to-stability/yield overlays (right).}
\end{figure}
\end{document}
